\documentclass{exam}%
\usepackage[T1]{fontenc}%
\usepackage[utf8]{inputenc}%
\usepackage{lmodern}%
\usepackage{textcomp}%
\usepackage{lastpage}%
\usepackage[top=1cm,bottom=0.5cm,left=1cm,right=1cm]{geometry}%
\usepackage{multicol}%
\usepackage{enumitem}%
\usepackage{fancyhdr}%
\usepackage{graphicx}%
\usepackage{amsmath}%
\usepackage{amssymb}%
\usepackage{fancybox}%
\usepackage{siunitx}%
%
\renewcommand{\questionlabel}{\thequestion.\hspace{0.5em}}%
\renewcommand{\choicelabel}{(\alph{choice})\hspace{0.3em}}%
\newcommand{\um}[2]{\SI[output-decimal-marker={,}]{#1}{#2}}%
\firstpageheader{}{Verifica di fisica}{}%
\runningfooter{}{}{Variante~\thevariant~/~\thepage}%
\newcounter{variant}%
%
\begin{document}%
\normalsize%
\setcounter{variant}{63}%
\noindent \textbf{\makebox[0.60\textwidth]{Nome e cognome:\enspace\hrulefill} \makebox[0.15\textwidth]{ Classe:\enspace\hrulefill} \makebox[0.20\textwidth]{ Data:\enspace\hrulefill}}%
\bigskip%
\noindent\textbf{Griglia Risposte (variante 63)}%
\begin{center}
\begin{tabular}{|c|c|c|c|c|c|c|c|c|c|}
\hline
1 & 2 & 3 & 4 & 5 & 6 & 7 & 8 & 9 & 10 \\ \hline
\rule{1cm}{0pt}\rule[-0.5em]{0pt}{1.5em} & \rule{1cm}{0pt}\rule[-0.5em]{0pt}{1.5em} & \rule{1cm}{0pt}\rule[-0.5em]{0pt}{1.5em} & \rule{1cm}{0pt}\rule[-0.5em]{0pt}{1.5em} & \rule{1cm}{0pt}\rule[-0.5em]{0pt}{1.5em} & \rule{1cm}{0pt}\rule[-0.5em]{0pt}{1.5em} & \rule{1cm}{0pt}\rule[-0.5em]{0pt}{1.5em} & \rule{1cm}{0pt}\rule[-0.5em]{0pt}{1.5em} & \rule{1cm}{0pt}\rule[-0.5em]{0pt}{1.5em} & \rule{1cm}{0pt}\rule[-0.5em]{0pt}{1.5em} \\ \hline
\end{tabular}
\end{center}%
\vspace{0.3em}%
\begin{center}
\begin{tabular}{|c|c|c|c|c|c|c|c|c|c|}
\hline
11 & 12 & 13 & 14 & 15 & 16 & 17 & 18 & 19 & 20 \\ \hline
\rule{1cm}{0pt}\rule[-0.5em]{0pt}{1.5em} & \rule{1cm}{0pt}\rule[-0.5em]{0pt}{1.5em} & \rule{1cm}{0pt}\rule[-0.5em]{0pt}{1.5em} & \rule{1cm}{0pt}\rule[-0.5em]{0pt}{1.5em} & \rule{1cm}{0pt}\rule[-0.5em]{0pt}{1.5em} & \rule{1cm}{0pt}\rule[-0.5em]{0pt}{1.5em} & \rule{1cm}{0pt}\rule[-0.5em]{0pt}{1.5em} & \rule{1cm}{0pt}\rule[-0.5em]{0pt}{1.5em} & \rule{1cm}{0pt}\rule[-0.5em]{0pt}{1.5em} & \rule{1cm}{0pt}\rule[-0.5em]{0pt}{1.5em} \\ \hline
\end{tabular}
\end{center}%
\vspace{1em}%
\begin{questions}%
\question Cosa postula il modello di Bohr riguardo all'emissione di radiazione da parte di un atomo?%
\vspace{0.2em}%
\begin{choices}%
\choice Un atomo emette radiazione solo se si trova in uno stato eccitato stazionario.%
\choice Un atomo emette radiazione continuamente mentre l'elettrone orbita attorno al nucleo.%
\choice Un atomo emette radiazione (un fotone) solo quando un elettrone salta da un'orbita permessa a un'altra orbita permessa di energia inferiore.%
\choice Un atomo emette radiazione solo quando viene ionizzato.%
\end{choices}%
\question Una radiazione di frequenza $f = 1.0 \times 10^{15} \, \text{Hz}$ colpisce un metallo con lavoro di estrazione $W = 2.0 \, \text{eV}$. Sapendo che $h \approx 6.63 \times 10^{-34} \, \text{J} \cdot \text{s}$ e $1 \, \text{eV} \approx 1.6 \times 10^{-19} \, \text{J}$, qual è circa l'energia cinetica massima $K_{max}$ degli elettroni emessi? (Suggerimento: calcola prima $hf$ in eV, $hf \approx 4.14 \, \text{eV}$)%
\vspace{0.2em}%
\begin{multicols}{4}%
\begin{choices}%
\choice $K_{max} \approx 2.0 \, \text{eV}$%
\choice $K_{max} \approx 4.14 \, \text{eV}$%
\choice $K_{max} \approx 6.14 \, \text{eV}$%
\choice $K_{max} \approx 2.14 \, \text{eV}$%
\end{choices}%
\end{multicols}%
\question Cosa dimostra in modo sorprendente l'esperimento della doppia fenditura con elettroni singoli?%
\vspace{0.2em}%
\begin{choices}%
\choice Che anche le singole particelle (elettroni) esibiscono un comportamento ondulatorio (interferenza), suggerendo che ogni elettrone "passa attraverso entrambe le fenditure" in senso quantistico.%
\choice Che il principio di indeterminazione non è valido.%
\choice Che la luce è composta da particelle (fotoni).%
\choice Che gli elettroni sono particelle classiche che seguono traiettorie ben definite.%
\end{choices}%
\question Identificare il prodotto mancante nel decadimento alfa dell'Uranio-238: $^{238}_{92}\text{U} \rightarrow X + \alpha$%
\vspace{0.2em}%
\begin{multicols}{4}%
\begin{choices}%
\choice $X = ^{234}_{88}\text{Ra}$ (Radio-234)%
\choice $X = ^{234}_{90}\text{Th}$ (Torio-234)%
\choice $X = ^{238}_{90}\text{Th}$ (Torio-238)%
\choice $X = ^{234}_{92}\text{U}$ (Uranio-234)%
\end{choices}%
\end{multicols}%
\question Il principio di indeterminazione è una conseguenza fondamentale:%
\vspace{0.2em}%
\begin{choices}%
\choice Del modello atomico di Bohr.%
\choice Degli errori sperimentali inevitabili negli strumenti di misura.%
\choice Della natura ondulatoria della materia (dualismo onda-corpuscolo) e dei limiti intrinseci alla misurazione nel mondo quantistico.%
\choice Della teoria della relatività di Einstein.%
\end{choices}%
\question Nel paradosso del gatto di Schrödinger, cosa rappresenta lo stato del gatto PRIMA che la scatola venga aperta, secondo un'interpretazione strettamente quantistica?%
\vspace{0.2em}%
\begin{choices}%
\choice Lo stato "gatto morto".%
\choice Uno stato indeterminato che non è né vivo né morto.%
\choice Lo stato "gatto vivo".%
\choice Una sovrapposizione quantistica degli stati "gatto vivo" e "gatto morto".%
\end{choices}%
\question Secondo la spiegazione di Einstein dell'effetto fotoelettrico, perché esiste una "frequenza di soglia" al di sotto della quale non vengono emessi elettroni, indipendentemente dall'intensità della luce?%
\vspace{0.2em}%
\begin{choices}%
\choice Perché l'energia del singolo fotone ($hf$) deve essere almeno pari al lavoro di estrazione ($W$) per liberare un elettrone.%
\choice Perché l'interazione tra luce e materia richiede un tempo minimo che dipende dalla frequenza.%
\choice Perché l'intensità della luce non è sufficiente a "scaldare" abbastanza gli elettroni.%
\choice Perché a basse frequenze la luce si comporta solo come un'onda.%
\end{choices}%
\question Nel range di energie tipico della radiodiagnostica (es. $30-150 \, \text{keV}$), quale interazione tra fotoni X e tessuti biologici (a basso Z) è generalmente dominante e più rilevante per la formazione dell'immagine?%
\vspace{0.2em}%
\begin{multicols}{2}%
\begin{choices}%
\choice Scattering di Rayleigh (coerente).%
\choice Effetto fotoelettrico.%
\choice Effetto Compton.%
\choice Produzione di coppie ($e^+/e^-$).%
\end{choices}%
\end{multicols}%
\question La "catastrofe ultravioletta" è un problema sorto nello studio della radiazione di corpo nero perché la fisica classica prevedeva:%
\vspace{0.2em}%
\begin{choices}%
\choice Un'intensità energetica nulla per lunghezze d'onda molto piccole.%
\choice Che l'intensità massima si spostasse verso il rosso (frequenze basse) all'aumentare della temperatura.%
\choice Che l'energia emessa fosse quantizzata fin dall'inizio.%
\choice Un'intensità energetica infinita per lunghezze d'onda molto piccole (alte frequenze).%
\end{choices}%
\question Quale tipo di decadimento radioattivo consiste nell'emissione di un nucleo di Elio ($^4_2\text{He}$)?%
\vspace{0.2em}%
\begin{multicols}{2}%
\begin{choices}%
\choice Decadimento Beta più ($\beta^+$)%
\choice Emissione Gamma ($\gamma$)%
\choice Decadimento Beta meno ($\beta^-$)%
\choice Decadimento Alfa ($\alpha$)%
\end{choices}%
\end{multicols}%
\question Come spiega il modello di Bohr l'emissione di luce a frequenze discrete (spettro a righe) da parte degli atomi?%
\vspace{0.2em}%
\begin{choices}%
\choice L'elettrone emette un fotone di energia definita ($E = hf$) quando salta da un'orbita permessa a energia superiore a una a energia inferiore.%
\choice L'elettrone emette luce continuamente mentre orbita, ma solo a certe frequenze.%
\choice Il nucleo atomico vibra emettendo fotoni.%
\choice Gli urti tra atomi eccitati producono lo spettro.%
\end{choices}%
\question Secondo l'esperimento mentale di Schrödinger, cosa determina il passaggio del gatto da uno stato di sovrapposizione a uno stato definito (vivo o morto)?%
\vspace{0.2em}%
\begin{choices}%
\choice Il decadimento dell'atomo radioattivo all'interno della scatola.%
\choice Il tempo trascorso dall'inizio dell'esperimento.%
\choice La volontà del gatto.%
\choice L'atto di osservazione o misurazione (apertura della scatola).%
\end{choices}%
\question Un isotopo radioattivo ha un tempo di dimezzamento di $T_{1/2} = 5 \, \text{giorni}$. Se inizialmente abbiamo $16 \, \text{mg}$ di questo isotopo, quanti milligrammi rimarranno dopo $20 \, \text{giorni}$?%
\vspace{0.2em}%
\begin{multicols}{4}%
\begin{choices}%
\choice $2 \, \text{mg}$%
\choice $8 \, \text{mg}$%
\choice $1 \, \text{mg}$%
\choice $4 \, \text{mg}$%
\end{choices}%
\end{multicols}%
\question Il nucleo di Deuterio ($^2_1\text{H}$) è formato da 1 protone ($m_p \approx 1.0073 \, \text{u}$) e 1 neutrone ($m_n \approx 1.0087 \, \text{u}$). La sua massa misurata è $m_D \approx 2.0141 \, \text{u}$. Qual è approssimativamente il difetto di massa $\Delta m$?%
\vspace{0.2em}%
\begin{multicols}{2}%
\begin{choices}%
\choice $\Delta m \approx (1.0073 + 1.0087) - 2.0141 = 0.0019 \, \text{u}$%
\choice $\Delta m \approx 2.0141 \, \text{u}$%
\choice $\Delta m \approx 2.0141 - (1.0073 + 1.0087) = -0.0019 \, \text{u}$%
\choice $\Delta m \approx 1.0073 + 1.0087 + 2.0141 \approx 4.0301 \, \text{u}$%
\end{choices}%
\end{multicols}%
\question Nell'effetto Compton, un fotone X interagisce con un elettrone libero (o debolmente legato). Cosa succede al fotone?%
\vspace{0.2em}%
\begin{choices}%
\choice Passa attraverso l'elettrone senza interagire.%
\choice Viene diffuso con una frequenza maggiore (lunghezza d'onda minore).%
\choice Viene diffuso (scatterato) con una frequenza minore (lunghezza d'onda maggiore).%
\choice Viene assorbito completamente dall'elettrone.%
\end{choices}%
\question La legge del decadimento radioattivo $N(t) = N_0 e^{-\lambda t}$ descrive:%
\vspace{0.2em}%
\begin{choices}%
\choice Il tempo di dimezzamento del campione.%
\choice Il numero di nuclei decaduti al tempo $t$.%
\choice L'attività del campione al tempo $t$.%
\choice Il numero $N(t)$ di nuclei radioattivi non ancora decaduti presenti al tempo $t$, partendo da $N_0$ nuclei al tempo $t=0$.%
\end{choices}%
\question In un esperimento Compton, un fotone X incide su un elettrone a riposo. La variazione della lunghezza d'onda ($\Delta \lambda = \lambda' - \lambda$) del fotone diffuso dipende dall'angolo di diffusione $\theta$. Quando è massima questa variazione?%
\vspace{0.2em}%
\begin{choices}%
\choice La variazione è indipendente dall'angolo $\theta$.%
\choice Quando l'angolo di diffusione è $\theta = 90^\circ$.%
\choice Quando l'angolo di diffusione è $\theta = 0^\circ$ (nessuna diffusione).%
\choice Quando l'angolo di diffusione è $\theta = 180^\circ$ (diffusione all'indietro).%
\end{choices}%
\question Completare la seguente reazione di decadimento beta meno ($\beta^-$): $^{14}_{6}\text{C} \rightarrow ? + e^- + \bar{\nu}_e$%
\vspace{0.2em}%
\begin{multicols}{4}%
\begin{choices}%
\choice $^{14}_{7}\text{N}$%
\choice $^{13}_{6}\text{C}$%
\choice $^{14}_{6}\text{C}$%
\choice $^{14}_{5}\text{B}$%
\end{choices}%
\end{multicols}%
\question Completare la seguente reazione di decadimento beta più ($\beta^+$) o cattura elettronica (EC), sapendo che il Fluoro-18 ($^{18}_{9}\text{F}$) può decadere $\beta^+$: $^{18}_{9}\text{F} \rightarrow ? + e^+ + \nu_e$%
\vspace{0.2em}%
\begin{multicols}{4}%
\begin{choices}%
\choice $^{18}_{10}\text{Ne}$%
\choice $^{18}_{8}\text{O}$%
\choice $^{19}_{9}\text{F}$%
\choice $^{17}_{9}\text{F}$%
\end{choices}%
\end{multicols}%
\question Come si calcola l'energia di legame ($E_B$) di un nucleo, noto il difetto di massa $\Delta m$?%
\vspace{0.2em}%
\begin{multicols}{4}%
\begin{choices}%
\choice $E_B = (\Delta m) / c^2$.%
\choice $E_B = m_{nucleo} c^2$.%
\choice $E_B = (\sum m_{costituenti}) c^2$.%
\choice $E_B = (\Delta m) c^2$.%
\end{choices}%
\end{multicols}%
\end{questions}%
\end{document}